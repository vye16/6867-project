\documentclass[11pt, twoside]{article}

\usepackage{graphicx}
\usepackage{amsmath, amssymb}
\usepackage{enumerate}
\usepackage{titleps}
\usepackage[top=1.25in,bottom=1in,right=1in,left=1in]{geometry}
\usepackage[parfill]{parskip}
\usepackage{titling}
\usepackage{hyperref}


\newpagestyle{ruled}
{\setfoot{}{\thepage}{} \footrule}
\pagestyle{ruled}


\setlength{\droptitle}{-4em}   % This is your set screw
\posttitle{\par\end{center}\vskip 0.5em}


\title{6.867 Project Proposal}
\date{}
\author {Vickie Ye and Alexandr Wang}


\begin{document}
\maketitle

For our final project, we are interested in exploring classification
techniques to classify facial emotions. Specifically, we would like
to explore two methods that have been used in facial recognition.
The first is to use principal component analysis to represent facial
images in the basis of the first principal components and classify the
resulting features with a multiclass classifier. In this part of the
project, we want to understand and implement PCA and apply
classification methods learned in class (neural networks, multiclass SVM).

The second method we want to explore is using convolutional neural networks
on the raw images. In this part of the project, we want to understand
and optimize the network structure of the CNN and compare its performance
to the other methods along various performance metrics (correctness, training
time, amount of training data required). We plan to use existing CNN libraries,
and refer to the methods and results of Lawrence et. al. \cite{Lawrence} and
Matsugu et. al. \cite{Matsugu}.

This project can be clearly divided into using PCA for feature
extraction from images for classification and using convolutional
neural networks for classification with little image preprocessing.
The project could then follow the rough timeline:
\begin{enumerate}
\item We first implement a baseline classifier using PCA and a neural network
classifier. We will evaluate the performance in terms of correctness, training
time, and amount of training data required.
\item We will then implement a multiclass SVM classifier with PCA features.
We will evaluate this method and compare it to the first method.
\item Finally, we will implement a convolutional neural network classifier
(using existing libraries). We expect this to be the most conceptually
difficult, because we will need to optimize the network structure to
prevent overfitting and have reasonable training times, among other
concerns.
\item We will finally evaluate the performance of the CNN along the
aforementioned metrics and compare to the other methods explored.
\end{enumerate}

Our goals in this project are to learn more about PCA and CNNs and their
application to image processing and classification. However we recognize
that there are some risks associated with this proposal, because we do
not have background in either. Our concerns with CNNs especially is the
amount of training data and training time they require for high performance.
Although we do not think it will be likely, in the event that training the
CNN to high performance becomes intractable, we can perform partial analysis
and comparison to the other listed methods. Because this project is a
comparison study of several approaches, we can be slightly more flexible
in the case of insurmountable difficulties in any one method.

\begin{thebibliography}{9}
\bibitem{Lawrence}
Lawrence, S.; Giles, L.; Tsoi, A. C.; Back, A. D. (1997)
``Face Recognition: A Convolutional Neural-Network Approach"
\textit{Neural Networks, IEEE transactions on} 8 (1):98-113
\bibitem{Matsugu}
Matsugu, M.; Mori, K.; Mitari Y.; Kaneda Y. (2003)
``Subject independent facial expression recognition with robust face detection using a convolutional neural network"
\textit{Neural Networks} 16 (5):555-559
\end{thebibliography}

\end{document}
